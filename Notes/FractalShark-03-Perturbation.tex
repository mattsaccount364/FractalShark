\section{Perturbation Rendering of the Mandelbrot Set}
\label{sec:perturbation-concept}

Rendering the Mandelbrot set (\cref{sec:mandelbrot-intro}) at extreme zoom levels presents a fundamental
numerical challenge. As the image is magnified, the parameters \(c\) associated
with individual pixels differ by increasingly small amounts, while the number
of iterations required to determine escape behavior typically grows. Accurately
tracking these orbits therefore requires both high numerical precision and a
large number of iterations, making naive per-pixel evaluation prohibitively
expensive.

Perturbation provides a mathematical and algorithmic framework
\cite{Martin2013Perturbation,HeilandAllen2013Perturbation,Wikibooks_Perturbation}
that addresses
this challenge by exploiting the strong coherence between nearby orbits. Rather
than evaluating each pixel independently at full precision, perturbation
separates the computation into a shared high-precision component and many
pixel-local low-precision components. This section introduces the perturbation
approach and explains how it enables efficient deep-zoom Mandelbrot rendering.

\subsection{Baseline iteration and its limitations}

The Mandelbrot set is defined by the iteration
\begin{equation}
z_{n+1} = z_n^2 + c, \qquad z_0 = 0,
\end{equation}
where \(c \in \mathbb{C}\) is the parameter associated with a pixel and the orbit
\(\{z_n\}\) is iterated until either escape is detected or a maximum iteration
count is reached.

At modest zoom levels, this iteration can be evaluated accurately using standard
floating-point arithmetic. At deep zoom levels, however, the differences between
nearby parameters \(c\) may be many orders of magnitude smaller than the values
of \(z_n\) encountered during iteration. In such regimes, rounding error and
loss of significance make fixed-precision arithmetic unreliable. While arbitrary
precision arithmetic can restore accuracy, applying it independently to every
pixel scales poorly.

\subsection{Coherence of nearby orbits}

A key observation underlying perturbation is that nearby parameters generate
orbits that remain close for many iterations. In a typical image tile, all pixel
parameters \(c\) lie within a small neighborhood of a central value. Their
orbits therefore share a common large-scale structure, differing only by small
corrections that grow gradually over time.

Perturbation makes this structure explicit by selecting a single
\emph{reference parameter} and expressing all nearby orbits relative to it. This
transforms the problem from one of many independent high-precision computations
into one of a single high-precision computation plus many low-precision updates.

\subsection{Reference orbit}

Let \(c_\star\) denote a reference parameter, typically chosen as the center of
the image region. Its orbit is computed using sufficiently high precision to
serve as a reliable baseline:
\begin{equation}
z_{n+1}^\star = (z_n^\star)^2 + c_\star, \qquad z_0^\star = 0.
\end{equation}
The sequence \(\{z_n^\star\}\) is stored and reused for all pixels within the
region. Because this orbit underpins the evaluation of many pixels, its
accuracy is critical.

\subsection{Perturbation as a mixed-precision decomposition}
\label{subsec:perturbation-mixed-precision}

For a nearby pixel parameter \(c = c_\star + \Delta c\), the orbit can be written
as a small deviation from the reference orbit:
\begin{equation}
z_n = z_n^\star + \Delta z_n.
\end{equation}
Substituting this expression into the Mandelbrot recurrence yields an exact
update rule for the deviation:
\begin{equation}
\Delta z_{n+1}
= 2 z_n^\star \Delta z_n + (\Delta z_n)^2 + \Delta c.
\label{eq:perturb-exact}
\end{equation}
No approximation has been introduced: as long as the reference orbit is computed
accurately, this formulation preserves the exact Mandelbrot dynamics.

This decomposition has an important numerical consequence. The reference orbit
\(z_n^\star\) typically grows in magnitude and requires arbitrary precision to be
represented accurately at deep zoom. In contrast, the deviation \(\Delta z_n\)
remains small for many iterations and can usually be evolved using much lower
precision arithmetic. Perturbation therefore reorganizes the computation into a
mixed-precision pipeline: expensive high-precision arithmetic is concentrated in
a single shared orbit, while the majority of per-pixel work is performed using
lower precision with significantly higher throughput.

\subsection{Real-valued formulation for implementation}

For practical implementation, particularly on GPUs, it is convenient to express
the perturbation update in real arithmetic. Writing
\begin{align*}
z_n^\star &= x_n^\star + i y_n^\star, \\
\Delta z_n &= \Delta x_n + i \Delta y_n, \\
\Delta c &= \Delta c_x + i \Delta c_y,
\end{align*}
and expanding \cref{eq:perturb-exact} yields
\begin{align}
\Delta x_{n+1}
&= 2(x_n^\star \Delta x_n - y_n^\star \Delta y_n)
   + (\Delta x_n^2 - \Delta y_n^2)
   + \Delta c_x, \\
\Delta y_{n+1}
&= 2(x_n^\star \Delta y_n + y_n^\star \Delta x_n)
   + 2 \Delta x_n \Delta y_n
   + \Delta c_y.
\end{align}
These equations map directly onto scalar arithmetic and can be evaluated
efficiently in parallel for many pixels.

\subsection{Reconstruction and escape testing}

Although perturbation evolves only the deviation \(\Delta z_n\), escape-time
rendering requires testing the magnitude of the full orbit value. At each
iteration, the current iterate is reconstructed as
\begin{equation}
z_n = z_n^\star + \Delta z_n,
\end{equation}
and the standard Mandelbrot bailout condition is applied:
\begin{equation}
|z_n|^2 \ge 4.
\end{equation}
The iteration at which this condition is first satisfied determines the pixel’s
escape time.

\subsection{Numerical stability and rebasing}

Perturbation remains efficient only while the deviation \(\Delta z_n\) remains
small relative to the reference orbit. As the deviation grows, numerical
cancellation and loss of significance may occur. Practical implementations
therefore employ \emph{rebasing} \cite{Zhuoran2023ReferenceCompression,HeilandAllen2022DeepZoom},
in which the current deviation
is folded into the reference state and perturbation continues relative to this
new baseline. This process preserves accuracy while maintaining numerical
stability.

\subsection{Relation to other acceleration techniques}

Perturbation preserves the exact dynamics of the Mandelbrot map and forms the
foundation of many advanced deep-zoom rendering techniques. It can be used by
itself (\cref{sec:perturb-only}) or combined with additional approximations, such as bilinear approximation
(\cref{sec:bilinear-approx}) or linear approximation (\cref{sec:la-v2-perturb}),
to skip multiple iterations at once. In all cases, perturbation provides
the conceptual framework that makes efficient deep-zoom Mandelbrot rendering
possible.

\paragraph{Terminology note: HDR scalar representation.}
In this work, the term \emph{HDR} refers to a \emph{high-dynamic-range} numeric
representation that pairs a fixed-precision mantissa (float or double) with an
explicitly managed integer exponent (see below for the full description).  When the GPU reference-orbit kernel or
perturbation pipeline needs lightweight control-flow decisions---periodicity
detection, escape testing, magnitude comparisons---it converts the full
arbitrary-precision state into an HDR value via a lossy reduction step
(\code{ToHDRFloat}).  The resulting HDR value preserves a wide exponent range,
sufficient to compare magnitudes spanning many orders of magnitude, but carries
only the fixed mantissa precision of the underlying float type rather than the
full arbitrary-precision mantissa.  The use of the term HDR here is purely
numerical and is unrelated to high-dynamic-range imaging.

\section{Perturbation-only Mandelbrot rendering (without linear approximation)}
\label{sec:perturb-only}

This kernel can render the Mandelbrot set using \emph{perturbation alone}, i.e.,
without taking any LA v2 linear-approximation steps. The same CUDA entry point
\code{mandel\_1xHDR\_float\_perturb\_lav2<IterType,T,SubType,Mode,PExtras>} is
used; the behavior is selected at compile time via \code{LAv2Mode}. In
particular, when \code{Mode} includes only the perturbation path (e.g.\
\code{LAv2Mode::PO}), the kernel skips the LA stage traversal and runs only the
perturbation loop against a stored reference orbit (\cref{sec:ref-orbit-calc}).

\subsection{Rendering objective}
\label{sec:perturb-only-goal}

The goal remains standard escape-time rendering for
\[
z_{n+1} = z_n^2 + c,\qquad z_0=0,
\]
with bailout \(|z|^2 \ge 4\). For each pixel, the kernel computes the parameter
\(c\), iterates until escape or \code{n\_iterations}, and stores the resulting
iteration count in \code{OutputIterMatrix[idx]}.

In perturbation rendering, the expensive high-precision orbit evaluation for
each pixel is avoided by reusing a \emph{reference orbit} computed at a
reference parameter \(c_\star\), then evolving only the \emph{delta orbit} for
nearby pixels.

\subsection{Reference orbit and delta formulation}
\label{sec:perturb-only-deltas}

Let the reference parameter be \(c_\star\), with stored reference orbit
\(\{z_n^\star\}\) satisfying
\[
z_{n+1}^\star = (z_n^\star)^2 + c_\star,\qquad z_0^\star=0.
\]
For a pixel parameter \(c\) near \(c_\star\), define the parameter delta
\[
\Delta c \stackrel{\mathrm{def}}{=} c - c_\star,
\]
and the orbit delta
\[
\Delta z_n \stackrel{\mathrm{def}}{=} z_n - z_n^\star.
\]
Substituting \(z_n = z_n^\star + \Delta z_n\) into the Mandelbrot recurrence
yields the \emph{exact} delta recurrence:
\begin{align}
\Delta z_{n+1}
&= (z_n^\star + \Delta z_n)^2 + (c_\star + \Delta c) - \left((z_n^\star)^2 + c_\star\right) \\
&= 2 z_n^\star\,\Delta z_n + (\Delta z_n)^2 + \Delta c.
\end{align}
Perturbation uses this recurrence directly: it is not a linearization. The work
per iteration is reduced because \(z_n^\star\) is fetched from storage rather
than recomputed in high precision.

\subsection{Pixel parameter delta \texorpdfstring{$\Delta c$}{Delta c}}
\label{sec:perturb-only-deltac}

For each pixel \((X,Y)\), the kernel constructs a delta parameter relative to a
chosen reference center (sign conventions incorporate the image mapping):
\begin{align}
\Delta c_x &= dx \cdot X - \code{centerX}, \\
\Delta c_y &= -dy \cdot Y - \code{centerY},
\end{align}
and stores this as \code{DeltaSub0} (with scalar components \code{DeltaSub0X},
\code{DeltaSub0Y}). The perturbation state starts at
\[
\Delta z_0 = 0,
\]
so \code{DeltaSubN} is initialized to zero.

\subsection{Perturbation recurrence used in the kernel}
\label{sec:perturb-only-update}

Writing the reference sample as \(z^\star = x^\star + i y^\star\) and the delta
as \(\Delta z = \Delta x + i \Delta y\), the delta recurrence can be expressed
in a factored form convenient for implementation:
\begin{equation}
\Delta z \leftarrow \Delta z \cdot (2 z^\star + \Delta z) + \Delta c.
\label{eq:perturb-factor}
\end{equation}
This identity is equivalent to
\(\Delta z_{n+1} = 2 z_n^\star \Delta z_n + (\Delta z_n)^2 + \Delta c\), because
\(\Delta z\cdot(2z^\star+\Delta z) = 2z^\star\Delta z + (\Delta z)^2\).

The kernel implements \cref{eq:perturb-factor} in real arithmetic by forming
the sums
\[
(2x^\star + \Delta x),\qquad (2y^\star + \Delta y),
\]
then updating:
\begin{align}
\Delta x' &= \Delta x\,(2x^\star + \Delta x) - \Delta y\,(2y^\star + \Delta y) + \Delta c_x, \\
\Delta y' &= \Delta x\,(2y^\star + \Delta y) + \Delta y\,(2x^\star + \Delta x) + \Delta c_y.
\end{align}
Note that both right-hand sides use the \emph{pre-update} values of
\(\Delta x\) and \(\Delta y\).  In the code, these originals are saved as
\code{DeltaSubNXOrig} and \code{DeltaSubNYOrig} before either component is
overwritten.  The intermediate quantities correspond to:
\begin{verbatim}
tempSum1 = 2*zy + DeltaSubNYOrig;  // (2 y^\star + \Delta y)
tempSum2 = 2*zx + DeltaSubNXOrig;  // (2 x^\star + \Delta x)
\end{verbatim}
followed by the real/imag updates. For extended or HDR types (\cref{sec:hdr-float}), the kernel routes
the same math through type-specialized implementations
(\code{T::custom\_perturb2} / \code{T::custom\_perturb3}) to keep the inner loop
tight and to enforce the type's reduction/normalization rules.

\subsection{Reconstructing the absolute orbit for escape testing}
\label{sec:perturb-only-reconstruct}

Perturbation evolves \(\Delta z_n\), but escape-time rendering requires a
bailout test on the absolute orbit \(z_n\). Each iteration reconstructs:
\[
z_n \approx z_n^\star + \Delta z_n,
\]
using the stored reference sample \(z_n^\star\) and the current delta. In the
kernel, this appears as:
\[
x = x^\star + \Delta x,\qquad y = y^\star + \Delta y.
\]
The bailout test is then performed on
\[
|z|^2 = x^2 + y^2,
\]
with the canonical threshold \(4\). For HDR and related types, the norm and the
comparison are performed on reduced values using specialized reduced
comparators to keep the escape decision stable at deep zoom.

\subsection{Reference index management and rebasing}
\label{sec:perturb-only-rebase}

The perturbation loop advances a reference-orbit index \code{RefIteration} in
lockstep with \code{iter}, fetching \(z^\star\) samples from \code{Perturb}. The
kernel includes a \emph{rebasing} mechanism that resets the delta
representation when it becomes ill-conditioned. Conceptually, if the delta
becomes comparable to or larger than the reconstructed orbit, the decomposition
\(z = z^\star + \Delta z\) stops being numerically advantageous. In that case,
the kernel folds the delta into the base by setting:
\[
\Delta z \leftarrow z,\qquad \text{and restart the reference index.}
\]
Operationally, the code compares the reconstructed orbit magnitude proxy
against the delta magnitude proxy, and also rebasing if the reference index
reaches the end of the stored orbit samples (to avoid out-of-range sampling).
After rebasing, perturbation continues from the new base representation.

This mechanism keeps the perturbation method usable across a wide range of
pixels and iteration depths while preserving the core rendering objective:
compute escape-time using a stable bailout on the reconstructed orbit.

\subsection{Using \code{LAv2Mode} to select perturbation-only execution}
\label{sec:perturb-only-mode}

The kernel is structured as two compile-time phases:
\begin{itemize}
\item an LA v2 phase guarded by \code{Mode == Full || Mode == LAO},
\item a perturbation phase guarded by \code{Mode == Full || Mode == PO}.
\end{itemize}
Therefore, perturbation-only rendering is achieved by instantiating the kernel
with a mode that includes only the perturbation path (e.g.\ \code{LAv2Mode::PO}).
In this configuration:
\begin{itemize}
\item \code{DeltaSub0} is computed from the pixel location,
\item \code{DeltaSubN} remains initialized to \(\Delta z_0 = 0\),
\item the kernel runs the perturbation loop, reconstructing \(z\) each step for
      bailout tests,
\item the final escape-time count is written to \code{OutputIterMatrix}.
\end{itemize}
This provides a complete Mandelbrot renderer based purely on reference-orbit
perturbation, without relying on any linear-approximation hierarchy.
After all pixels are computed, the antialiasing and reduction passes
(\cref{subsec:antialiasing,subsec:reduction}) extract the final colour
image and iteration statistics.


\section{Approximation-Based Orbit Acceleration}
\label{sec:approx-accel}

At deep zoom, the cost of iterating high-precision types for every pixel can be
dominant. The remaining components described below are still in service of the
same rendering goal: compute escape-time for \(z_{n+1}=z_n^2+c\), but by reusing
a \emph{reference orbit} and evolving \emph{deltas} for nearby pixels.

\subsection{Bilinear approximation (BLA) for orbit deltas}
\label{sec:bilinear-approx}

Bilinear approximation \cite{Thompson2023BLA} precomputes composable linear
maps that advance the perturbation delta by multiple iterations at once,
avoiding per-step evaluation when the linearization error remains small.

\subsubsection{Reference orbit and delta formulation}
Let the reference parameter be \(c_\star\) with orbit
(see \cref{sec:perturbation-concept} for the general perturbation framework):
\[
z_{n+1}^\star = (z_n^\star)^2 + c_\star,\qquad z_0^\star=0.
\]
For a nearby pixel parameter \(c = c_\star + \Delta c\), define:
\[
\Delta z_n \stackrel{\mathrm{def}}{=} z_n - z_n^\star,\qquad
\Delta c \stackrel{\mathrm{def}}{=} c - c_\star.
\]
Expanding the recurrence gives:
\begin{align}
\Delta z_{n+1}
&= (z_n^\star + \Delta z_n)^2 + (c_\star + \Delta c) - \left((z_n^\star)^2 + c_\star\right)\\
&= 2 z_n^\star\,\Delta z_n + (\Delta z_n)^2 + \Delta c.
\end{align}
When \(\Delta z_n\) remains small, the quadratic term can be neglected,
yielding a linearized update:
\begin{equation}
\Delta z_{n+1} \approx A_n\,\Delta z_n + B_n\,\Delta c,\qquad
A_n = 2z_n^\star,\quad B_n = 1.
\end{equation}
BLA generalizes this into precomputed multi-step maps that \emph{jump} multiple
iterations while maintaining an explicit validity bound.

\subsubsection{What \code{BLA<T>} stores}
A \code{BLA<T>} instance stores two complex coefficients:
\[
A = A_x + iA_y,\qquad B = B_x + iB_y,
\]
plus:
\begin{itemize}
  \item \code{r2}: a squared-radius validity bound used during lookup,
  \item \code{l}: the number of Mandelbrot iterations summarized by this step.
\end{itemize}
These objects exist to accelerate \emph{escape-time evaluation} by evolving
\(\Delta z\) cheaply for many pixels, then reconstructing \(z \approx z^\star +
\Delta z\) to perform bailout checks consistent with Mandelbrot rendering.

\subsubsection{Applying a step: complex multiply-add}
The method \code{getValue} (taking arguments \code{RealDeltaSubN},
\code{ImagDeltaSubN}, \code{RealDeltaSub0}, \code{ImagDeltaSub0})
applies:
\[
\Delta z \leftarrow A\,\Delta z + B\,\Delta c,
\]
expanded into real arithmetic:
\begin{align}
\Re(\Delta z') &= A_x \Re(\Delta z) - A_y \Im(\Delta z) + B_x \Re(\Delta c) - B_y \Im(\Delta c), \\
\Im(\Delta z') &= A_x \Im(\Delta z) + A_y \Re(\Delta z) + B_x \Im(\Delta c) + B_y \Re(\Delta c).
\end{align}

\subsubsection{Composing steps to build longer jumps}
If a first step maps \(\Delta z \mapsto A_1\Delta z + B_1\Delta c\) and a second maps
\(\Delta z' \mapsto A_2\Delta z' + B_2\Delta c\), the composition is:
\[
A_{\text{new}} = A_2 A_1,\qquad B_{\text{new}} = A_2 B_1 + B_2.
\]
In the code (\code{BLA.h}), these two steps are passed as the parameters
\code{x} and \code{y} to \code{getNewA} and \code{getNewB}, corresponding to
step~1 and step~2 respectively.
This supports a hierarchy of step sizes (often powers of two) for quickly
advancing delta orbits while rendering the escape-time field.

\subsubsection{GPU lookup: selecting a valid aligned step}
A GPU-side helper such as \code{GPU\_BLAS} stores the hierarchy and selects a
step that is both \emph{aligned} with the current iteration index and
\emph{valid} under the current bound check (typically comparing a computed
squared-magnitude proxy \code{z2} against \code{r2}). When a step is valid, the
renderer can advance the orbit by \code{l} iterations at a cost far below
performing \code{l} full high-precision Mandelbrot updates.

\subsection{LA v2 linear approximation with perturbation (HDR kernel)}
\label{sec:la-v2-perturb}

This kernel family combines staged linear-approximation steps \cite{Imagina} with a
perturbation finisher loop against a stored reference orbit
(see \cref{sec:perturb-only} for the perturbation-only variant). The rendering
objective remains escape-time Mandelbrot evaluation; the kernel accelerates
that evaluation by evolving deltas and periodically reconstructing \(z\) to
perform bailout checks.

\subsubsection{Parameter delta per pixel}
Each pixel constructs a parameter offset \(\Delta c\) relative to a selected
reference center (sign conventions may incorporate the image \(Y\)-flip):
\begin{align}
\Delta c_x &= dx\cdot X - \code{centerX}, \\
\Delta c_y &= -dy\cdot Y - \code{centerY},
\end{align}
and packs \(\Delta c=\Delta c_x+i\Delta c_y\) into \code{DeltaSub0}.

\subsubsection{Delta state and reconstruction}
The kernel maintains \(\Delta z_n\) in \code{DeltaSubN} and reconstructs an
absolute orbit estimate using the stored reference orbit sample
\(z_j^\star\):
\[
z \approx z_j^\star + \Delta z.
\]
Escape-time rendering then proceeds by applying approximation steps when valid,
or performing perturbation updates otherwise, while periodically testing the
bailout condition on the reconstructed \(z\).

\subsubsection{Perturbation update}
Given a reference sample \(z^\star=x^\star+iy^\star\) and \(\Delta z=\Delta
x+i\Delta y\), the perturbation form is:
\[
\Delta z \leftarrow \Delta z\cdot(2z^\star+\Delta z) + \Delta c.
\]
In real arithmetic, with temporaries corresponding to \((2x^\star+\Delta x)\)
and \((2y^\star+\Delta y)\), this yields:
\begin{align}
\Delta x' &= \Delta x\,(2x^\star+\Delta x) - \Delta y\,(2y^\star+\Delta y) + \Delta c_x, \\
\Delta y' &= \Delta x\,(2y^\star+\Delta y) + \Delta y\,(2x^\star+\Delta x) + \Delta c_y,
\end{align}
where both right-hand sides use the pre-update values of \(\Delta x\) and
\(\Delta y\) (the code saves these as \code{DeltaSubNXOrig} and
\code{DeltaSubNYOrig}).

\subsubsection{Escape test}
After updating \(\Delta z\), reconstruct \(z\approx z^\star+\Delta z\) and test:
\[
|z|^2 = x^2+y^2 \ge 4 \quad\Rightarrow\quad \text{escaped.}
\]
For HDR/expanded types, norms and comparisons are performed using reduced values
and specialized reduced comparators, preserving meaningful bailout decisions at
deep zoom.



