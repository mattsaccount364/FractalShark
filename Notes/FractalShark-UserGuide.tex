\section{User Interface and Popup Menu Guide}
\label{sec:ui-popup-menu}

\FractalShark{} can be driven \emph{entirely} from the keyboard and mouse, with the
right-click popup menu acting as the primary discovery and configuration
surface.  Day-to-day navigation (zooming, recentering, going back, iteration
adjustment, palette cycling, and several recomputation paths) is also exposed
through single-key shortcuts, so common workflows do not require opening the
menu.

The popup menu is context-sensitive and rebuilt dynamically each time it is
shown, so that checked, enabled, and disabled states accurately reflect the
current application settings and rendering state.  The menu can be invoked by
mouse (\textbf{right-click}) or from the keyboard via the standard Windows
context-menu invocation (\textbf{Shift+F10} / \textbf{Menu} key).

\subsection{Basic Interaction}

On startup, \FractalShark{} displays a brief splash screen (randomly chosen
from a set of built-in images) while initializing the renderer.

The most common interactions are:
\begin{itemize}
  \item \textbf{Right-click}: Open the popup menu at the mouse cursor.
  \item \textbf{Left-click and drag}: Zoom into a rectangular region.
        By default the zoom rectangle preserves the window aspect ratio;
        holding \texttt{Shift} while dragging allows a free-form
        (non-square) selection.
  \item \textbf{Mouse wheel}: Scroll forward to zoom in at the cursor;
        scroll backward to zoom out from the center.
  \item \textbf{Keyboard shortcuts}:
    \begin{itemize}
      \item \texttt{z}: Zoom in at the mouse cursor.
      \item \texttt{Shift+Z}: Zoom out slightly.
      \item \texttt{b}: Return to the previous view.
      \item \texttt{- / =}: Decrease or increase the iteration limit.
    \end{itemize}
\end{itemize}

Resizing the window automatically resets the renderer dimensions and triggers
a recalculation of the fractal to match the new window size.

\subsection{Menu Structure and Behavior}

The popup menu is organized into logical groups and submenus. Items fall into
three distinct behavioral categories:

\begin{description}
  \item[Radio groups] Mutually exclusive choices where exactly one option is
  active at a time (for example, selecting a render algorithm or GPU antialiasing level).
  \item[Toggles] Binary on/off options that switch between two states each time
  they are selected (for example, toggling repainting or full-screen mode).
  \item[Commands] Immediate actions that execute when selected and do not
  maintain a checked state.
\end{description}

Radio groups reflect the configuration in use, with the currently active option
checked. Selecting a different option automatically deselects the previous
choice and triggers any necessary reconfiguration.

\subsection{Top-Level Menu}

\begin{itemize}
  \item \emph{Show Help} (command): Displays an on-screen help dialog summarizing keyboard shortcuts.

  \item \emph{Navigate} (submenu):
    \begin{itemize}
      \item \emph{Back} (command): Navigate back to the previous view in history.
      \item \emph{Center View Here} (command): Centers the current view on the
      point under the cursor.
      \item \emph{Zoom In Here} (command): Zooms in toward the point under the cursor.
      \item \emph{Zoom Out} (command): Zooms out from the current view.
      \item \emph{Autozoom Feature} (command): See \cref{par:autozoom} for details.
      \item \emph{Autozoom Default} (command): See \cref{par:autozoom} for details.
      \item \emph{Feature Finder} (submenu): Locates nearby mini-Mandelbrot
      nuclei (periodic points) using Newton's method and optionally zooms to
      them.  Three evaluation back-ends are available, each in a single-point
      and a whole-screen scan variant.  See
      \cref{sec:feature-finder} for the underlying algorithm.
        \begin{itemize}
          \item \emph{Direct} (\texttt{n}) (command): Runs the feature finder at
          the mouse cursor using direct iteration.  Best for shallow zooms where
          no perturbation data is needed.
          \item \emph{DirectScan} (\texttt{N}) (command): Runs direct-iteration
          feature finding on a $12\times12$ grid across the entire viewport,
          locating multiple features at once.
          \item \emph{PT} (\texttt{m}) (command): Runs the feature finder at the
          mouse cursor using perturbation theory (\cref{sec:perturbation-concept}).
          Requires a reference orbit and works at deeper zooms than Direct.
          \item \emph{PTScan} (\texttt{M}) (command): Runs perturbation-theory
          feature finding on a $12\times12$ grid across the viewport.
          \item \emph{LA} (\texttt{,}) (command): Runs the feature finder at the
          mouse cursor using linear approximation (\cref{sec:la-v2-perturb}).
          Works at extreme zoom depths where PT alone would be too slow.
          \item \emph{LAScan} (\texttt{<}) (command): Runs linear-approximation
          feature finding on a $12\times12$ grid across the viewport.
          \item \emph{Zoom to Found Feature} (\texttt{.}) (command): Zooms into
          the most recently found feature.  If the feature has only been located
          at low precision, a high-precision GMP refinement pass is performed
          automatically before zooming.
          \item \emph{Clear All Found Features} (\texttt{>}) (command): Removes
          all found-feature overlays (lines and circles drawn on the fractal).
        \end{itemize}
      When a feature is found, a line is drawn from the search origin to the
      detected nucleus so you can see where it is.  Scan modes are useful for
      surveying an entire view at a glance; single-point modes are faster and
      target the area near your cursor.
    \end{itemize}

  \item \emph{Built-In Views} (submenu): Loads a predefined view/location used
  for demonstrations, debugging, or performance testing.  Up to 31 views are
  defined in the source (\texttt{FractalViewPresets.cpp}).
    \begin{itemize}
      \item \emph{Help} (command): Pops up a message box that simply says this
      submenu contains built-in test views.
      \item \emph{Standard View} (command): Loads the default starting view: the
      classic Mandelbrot.
      \item \emph{\#1 -- \#31} (commands): Loads a specific predefined
      test view. Many entries encode expected precision limits, kernel behavior,
      or known failure cases in their label text.  Several views require intense
      computation and take a long time to render.  View \#27, for example, is a
      period ~28 billion point that takes hours to render and requires reference
      compression even with 128GB of RAM.  Exercise caution when selecting deep-zoom
      views, as they may cause long delays or high memory usage.

    \end{itemize}

  \item \emph{Recalculate, Reuse Reference} (command): Forces an immediate
  re-render using the current settings, reusing the existing reference orbit if
  possible.

  \item \emph{Toggle Repainting}: Enables or disables progressive
  repainting while a render is in progress. When disabled, no recalculation
  takes place on window resize or related, which can be helpful if you want a
  new window size but don't want to trigger a re-render.
  \item \emph{Toggle Window Size}: Toggles between windowed and
  full-screen behavior.  The window stays at the same aspect ratio by default.
  \item \emph{Toggle Window Size (Square)}: Toggles an alternate sizing
  mode constrained to a square aspect ratio.
  \item \emph{Minimize Window} (command): Minimizes the application window.

  \item \emph{Antialiasing} (radio group submenu): Selects the antialiasing
  level used for rendering.
    \begin{itemize}
      \item \emph{1x (fast)}
      \item \emph{4x}
      \item \emph{9x}
      \item \emph{16x (better quality)}
    \end{itemize}

  Note that antialiasing as used here is not traditional GPU antialiasing.
  Instead, 4, 9, or 16 samples are explicitly computed per pixel and averaged
  together either on the CPU or GPU.  This approach is known as
  \emph{supersampling} and produces high-quality results at the cost of
  increased computation time and memory usage.  It's a naive approach---in
  principle, more sophisticated techniques could be implemented in the future,
  such as edge-aware sampling or adaptive sampling.

  \item \emph{Choose Render Algorithm} (radio group submenu): Selects the core
  Mandelbrot rendering strategy. This is a radio group: selecting one algorithm
  automatically deselects the others. Available choices may include CPU-based
  reference and linear-approximation renderers, multiple CUDA-based GPU
  renderers, and high-dynamic-range and experimental kernels. On systems without
  a working CUDA setup, CPU renderers can be selected as a fallback. In the
  event of a CUDA initialization failure, switching to a CPU algorithm followed
  by a recalculation may restore a usable image.

    \begin{itemize}
      \item \emph{Help} (command): Displays help for algorithm selection and
      related concepts (e.g., what LAv2 (\cref{sec:la-v2-perturb}), BLA
      (\cref{sec:bilinear-approx}), perturbation (\cref{sec:perturbation-concept}),
      and reference compression (\cref{subsec:ref-orbit-compression-reuse}) mean,
      and when to use each).

      \item \emph{Auto (Default)} (radio): Automatically selects the
      default/recommended renderer.  Recommended.

      \emph{Note}: this option exhibits potentially-confusing behavior.  When
      selected, it does not lock in a specific algorithm; rather, it allows the
      system to choose the most appropriate algorithm based on current view
      parameters. The radio button for "Auto" will not be checked; instead, the
      actual algorithm in use will be checked.  Thus, selecting \emph{Auto} may
      appear to have no effect but if you navigate the submenus you'll find the
      algorithm currently in use is checked.  This design allows users to see
      what algorithm is active.

      At present, only three algorithms are eligible for automatic selection:
      \begin{itemize}
        \item \emph{1x32 GPU}: The default low-zoom GPU linear-approximation
              renderer.  No perturbation or linear approximation is used.
        \item \emph{1x32 GPU -- Perturbation Only}: The default
              perturbation-only GPU renderer (\cref{sec:perturb-only}) for
              moderate zooms uses perturbation but no linear approximation.
        \item \emph{1x32 GPU -- LAv2}: A bit deeper, no high-dynamic range
              floats but does use linear-approximation renderer and
              perturbation.
        \item \emph{HDRx32 GPU -- LAv2}: Arbitrary depth (\cref{sec:la-v2-perturb,sec:hdr-float}), combines all known
              algorithmic optimizations.  This one can result in rendering
              artifacts at some deep locations due to numeric precision
              limitations.  "Auto" mode can't tell, so if you see artifacts, try
              HDRx2x32 or HDRx64 variants instead.  View \#26 is an example of a
              point that fails with HDRx32 but works correctly with HDRx64.

              See \cref{fig:auto-renderer-artifact} for an example of
              artifacts that can arise when using HDRx32 at extreme depths.
      \end{itemize}

      \item \emph{LA Parameters} (submenu): Controls linear-approximation (LA) configuration and tuning.
        \begin{itemize}
          \item \emph{Multithreaded (Default)} (radio): Uses multithreaded LA processing.
          \item \emph{Single Threaded} (radio): Uses single-threaded LA processing.
          \item \emph{Max Accuracy (Default)} (command): \textit{[TODO: unclear behavior/description]}
          \item \emph{Max Performance (Accuracy Loss)} (command): \textit{[TODO: unclear behavior/description]}
          \item \emph{Min Memory} (command): \textit{[TODO: unclear behavior/description]}
        \end{itemize}

      \item \emph{CPU-Only} (submenu): CPU renderers and CPU-driven perturbation variants.
        \begin{itemize}
          \item \emph{Very High Precision CPU} (radio)
          \item \emph{1x64 CPU} (radio)
          \item \emph{HDRx32 CPU} (radio)
          \item \emph{HDRx64 CPU} (radio)

          \item \emph{1x64 CPU -- Perturbation BLA} (radio)
          \item \emph{HDRx32 CPU -- Perturbation BLA} (radio)
          \item \emph{HDRx64 CPU -- Perturbation BLA} (radio)

          \item \emph{HDRx32 CPU -- Perturbation LAv2} (radio)
          \item \emph{HDRx64 CPU -- Perturbation LAv2} (radio)
          \item \emph{HDRx32 RC CPU -- Perturbation LAv2} (radio)
          \item \emph{HDRx64 RC CPU -- Perturbation LAv2} (radio)
        \end{itemize}

      \item \emph{Low-Zoom Depth} (submenu): GPU renderers intended for relatively modest zoom depths and lower precision requirements.
        \begin{itemize}
          \item \emph{Iteration Precision} (radio group submenu): Selects the iteration precision mode used by low-zoom GPU paths.
            \begin{itemize}
              \item \emph{4x (fast)} (radio)
              \item \emph{3x} (radio)
              \item \emph{2x} (radio)
              \item \emph{1x (better quality)} (radio)
            \end{itemize}

          \item \emph{1x32 GPU} (radio)
          \item \emph{2x32 GPU} (radio)
          \item \emph{4x32 GPU} (radio)
          \item \emph{1x64 GPU} (radio)
          \item \emph{2x64 GPU} (radio)
          \item \emph{4x64 GPU} (radio)
          \item \emph{HDRx32 GPU} (radio)
        \end{itemize}

      \item \emph{Scaled} (submenu): Perturbation renderers using a scaled formulation.
        \begin{itemize}
          \item \emph{1x32 GPU -- Perturbation Scaled} (radio)
          \item \emph{2x32 GPU -- Perturbation Scaled (broken)} (radio)
          \item \emph{HDRx32 GPU -- Perturbation Scaled} (radio)
        \end{itemize}

      \item \emph{Bilinear Approximation V1} (submenu): Perturbation renderers using the original bilinear approximation approach.
        \begin{itemize}
          \item \emph{1x64 GPU -- Perturbation BLA} (radio)
          \item \emph{HDRx32 GPU -- Perturbation BLA} (radio)
          \item \emph{HDRx64 GPU -- Perturbation BLA} (radio)
        \end{itemize}

      \item \emph{LA Only (for testing)} (submenu): Test configurations that run only the linear-approximation portion of the pipeline.
        \begin{itemize}
          \item \emph{1x32 GPU -- LAv2 -- LA only} (radio)
          \item \emph{2x32 GPU -- LAv2 -- LA only} (radio)
          \item \emph{1x64 GPU -- LAv2 -- LA only} (radio)
          \item \emph{HDRx32 GPU -- LAv2 -- LA only} (radio)
          \item \emph{HDRx2x32 GPU -- LAv2 -- LA only} (radio)
          \item \emph{HDRx64 GPU -- LAv2 -- LA only} (radio)
        \end{itemize}

      \item \emph{Perturbation Only} (submenu): Test configurations that run only the perturbation portion of the pipeline.
        \begin{itemize}
          \item \emph{1x32 GPU -- Perturb only} (radio)
          \item \emph{2x32 GPU -- Perturb only} (radio)
          \item \emph{1x64 GPU -- Perturb only} (radio)
          \item \emph{HDRx32 GPU -- Perturb only} (radio)
          \item \emph{HDRx2x32 GPU -- Perturb only} (radio)
          \item \emph{HDRx64 GPU -- Perturb only} (radio)
        \end{itemize}

      \item \emph{Reference Compression} (submenu): Algorithms using reference-orbit compression as part of the perturbation pipeline.
        \begin{itemize}
          \item \emph{1x32 GPU -- RC LAv2} (radio)
          \item \emph{2x32 GPU -- RC LAv2} (radio)
          \item \emph{1x64 GPU -- RC LAv2} (radio)
          \item \emph{HDRx32 GPU -- RC LAv2} (radio)
          \item \emph{HDRx2x32 GPU -- RC LAv2} (radio)
          \item \emph{HDRx64 GPU -- RC LAv2} (radio)

          \item \emph{Perturbation Only} (submenu): Reference-compressed perturbation-only variants.
            \begin{itemize}
              \item \emph{1x32 GPU -- RC Perturb Only} (radio)
              \item \emph{2x32 GPU -- RC Perturb Only} (radio)
              \item \emph{1x64 GPU -- RC Perturb Only} (radio)
              \item \emph{HDRx32 GPU -- RC Perturb Only} (radio)
              \item \emph{HDRx2x32 GPU -- RC Perturb Only} (radio)
              \item \emph{HDRx64 GPU -- RC Perturb Only} (radio)
            \end{itemize}

          \item \emph{LA Only} (submenu): Reference-compressed LA-only variants.
            \begin{itemize}
              \item \emph{1x32 GPU -- RC LAv2} (radio)
              \item \emph{2x32 GPU -- RC LAv2} (radio)
              \item \emph{1x64 GPU -- RC LAv2} (radio)
              \item \emph{HDRx32 GPU -- RC LAv2} (radio)
              \item \emph{HDRx2x32 GPU -- RC LAv2} (radio)
              \item \emph{HDRx64 GPU -- RC LAv2} (radio)
            \end{itemize}
        \end{itemize}

      \item \emph{LAv2 (Full Pipeline)} (radio entries): Main LAv2 renderers combining LA, perturbation, and reference handling.
        \begin{itemize}
          \item \emph{1x32 GPU -- LAv2} (radio)
          \item \emph{2x32 GPU -- LAv2} (radio)
          \item \emph{1x64 GPU -- LAv2} (radio)
          \item \emph{HDRx32 GPU -- LAv2} (radio)
          \item \emph{HDRx2x32 GPU -- LAv2} (radio)
          \item \emph{HDRx64 GPU -- LAv2} (radio)
        \end{itemize}

      \item \emph{Tests} (submenu): Development and diagnostic test actions.
        \begin{itemize}
          \item \emph{Run Basic Test (saves files in local dir)} (command)
          \item \emph{Run View \#27 test} (command)
        \end{itemize}

    \end{itemize}

  \item \emph{Iterations} (submenu): Adjusts the iteration limit and iteration counter width.
    \begin{itemize}
      \item \emph{Default Iterations} (command): Resets iteration count to defaults.
      \item \emph{+1.5x} (command): Multiplies iteration limit by 1.5.
      \item \emph{+6x} (command): Multiplies iteration limit by 6.
      \item \emph{+24x} (command): Multiplies iteration limit by 24.
      \item \emph{Decrease Iterations} (command): Decreases the current iteration limit.
      \item \emph{32-Bit Iterations} (radio): Uses 32-bit iteration counters (lower memory, smaller range).
      \item \emph{64-Bit Iterations} (radio): Uses 64-bit iteration counters (larger range).
    \end{itemize}

  \item \emph{Perturbation} (radio group submenu): Controls perturbation rendering and related reference-orbit management. \FractalShark{} supports perturbation-based rendering driven by a high-precision reference orbit. Menu options in this area control how the reference orbit is computed and reused. GPU-accelerated reference orbits are intended for extremely deep zooms and may be slower than CPU computation at modest precision levels. They are marked as experimental and should be used accordingly.
    \begin{itemize}
      \item \emph{Clear Perturbation References -- All} (command): Clears all cached perturbation references.
      \item \emph{Clear Perturbation References -- Med} (command): Clears medium-class cached references.
      \item \emph{Clear Perturbation References -- High} (command): Clears high-class cached references.
      \item \emph{Show Perturbation Results} (command): Intended to display
      diagnostic overlay information for perturbation state.  Currently
      a stub (prints a TODO to the debug console).

      \item \emph{Auto (default)} (radio): Automatically selects the perturbation
      algorithm based on current conditions.  Recommended for most use.
      \item \emph{Single Thread (ST)} (radio): Uses a single-threaded
      reference-orbit computation with no periodicity checking.
      \item \emph{Multi Thread (MT)} (radio): Uses a multithreaded
      reference-orbit computation with no periodicity checking.
      \item \emph{ST + Periodicity} (radio): Single-threaded reference-orbit
      computation with periodicity checking to detect convergence early.
      \item \emph{MT2 + Periodicity} (radio): Multithreaded reference-orbit
      computation (generation~2) with periodicity checking.
      \item \emph{MT2 + Periodicity + Perturb ST} (radio): MT2 reference orbit
      with a single-threaded perturbation pass for the high-resolution layer.
      \item \emph{MT2 + Periodicity + Perturb MT v1--v4} (radio entries):
      MT2 reference orbit with multithreaded perturbation.  Versions v1--v3
      represent different threading/scheduling strategies for the per-pixel
      perturbation pass; v4 is marked as broken.
      \item \emph{MT5 + Periodicity} (command): A newer multithreaded variant
      (generation~5) with periodicity checking.  Only enabled when perturbation
      support is available.

      \item \emph{GPU-Accelerated (see README)} (command): Selects or enables a GPU-accelerated perturbation path. \textit{[TODO: clarify whether this is a mode toggle, a mode selection, or a one-shot action]}

      \item \emph{Clear and Reload Reference Orbits} (command): Clears reference-orbit state and reloads it from persistent storage.
      \item \emph{Save Reference Orbits} (command): Saves reference-orbit state to persistent storage.
    \end{itemize}

  \item \emph{Palette Color Depth} (submenu): Controls palette selection, palette bit depth, and optional palette rotation. Some entries may be enabled only when palette rotation support is available.
    \begin{itemize}
      \item \emph{Basic} (radio): \textit{[TODO: clarify palette type ``Basic'']}
      \item \emph{Default} (radio)
      \item \emph{Patriotic} (radio)
      \item \emph{Summer} (radio)
      \item \emph{Random} (radio)

      \item \emph{Create Random Palette} (command): Generates a new random palette.

      \item \emph{5-bit} (radio)
      \item \emph{6-bit} (radio)
      \item \emph{8-bit} (radio)
      \item \emph{12-bit} (radio)
      \item \emph{16-bit} (radio)
      \item \emph{20-bit} (radio)

      \item \emph{Palette Rotation} (command): \textit{[TODO: clarify whether this is a toggle or a one-shot action; enabled only when rotation is supported]}
    \end{itemize}

  \item \emph{Memory Management} (submenu): Controls memory-limit policy and whether reference orbit data is automatically saved.
    \begin{itemize}
      \item \emph{Enable Auto-Save Orbit (delete when done, default)} (radio): Automatically saves orbit data temporarily and deletes it when no longer needed.
      \item \emph{Enable Auto-Save Orbit (keep files)} (radio): Automatically saves orbit data and retains files on disk.
      \item \emph{Disable Auto-Save Orbit} (radio): Disables automatic orbit saving.

      \item \emph{Remove Memory Limits} (radio): Disables memory limiting behavior.
      \item \emph{Leave max of (1/2*ram, 8GB) free (default)} (radio): Constrains allocation to preserve system memory headroom.
    \end{itemize}

  \item \emph{Show Rendering Details} (command): Displays current rendering
  status and diagnostics, including position coordinates (real, imaginary,
  zoom level), the active render algorithm, and perturbation/LA state.
  The information is shown in a message box and also copied to the clipboard.

  \item \emph{Save} (submenu): Output and benchmarking commands.
    \begin{itemize}
      \item \emph{Save Location...} (command): Saves the current location/view parameters.
      \item \emph{Save High Res Bitmap} (command): Saves a high-resolution bitmap render. \textit{[TODO: clarify how resolution is selected]}
      \item \emph{Save Iterations as Text} (command): Saves iteration counts in a text format.
      \item \emph{Save Bitmap Image} (command): Saves the current bitmap image.

      \item \emph{Save Reference Orbit as Text} (command): Saves the reference orbit in text form.
      \item \emph{Save Compressed Orbit as Text (simple)} (command)
      \item \emph{Save Compressed Orbit as Text (max)} (command)
      \item \emph{Save Compressed Orbit (Imagina/max)} (command): Saves the
      reference orbit in Imagina's file format with maximum compression,
      suitable for import into Imagina or later reload via
      \emph{Load~$\to$~Load Imagina Orbit}.

      \item \emph{Benchmark (5x, full recalc)} (command): Runs 5 full
      recalculation passes using all perturbation result types and reports
      overall and per-pixel rendering times in milliseconds.
      \item \emph{Benchmark (5x, intermediate only)} (command): Runs 5
      recalculation passes using only the medium-resolution perturbation
      results, which isolates intermediate-stage performance.

      \item \emph{Diff Imagina Orbits (choose two)} (command): Opens two
      file-chooser dialogs to select a pair of Imagina-format orbit files, then
      compares them element-by-element and reports any differences.  Useful for
      verifying that two orbit computations produce equivalent results.
    \end{itemize}

  \item \emph{Load} (submenu): Load view parameters and reference orbits.
    \begin{itemize}
      \item \emph{Load Location...} (command): Loads a previously saved location/view.
      \item \emph{Enter Location} (command): Manually enters a location/view
      via a dialog where you type the real part, imaginary part, and zoom level.
      \item \emph{Load Imagina Orbit (Match)...} (command): Loads an
      Imagina-format orbit file and matches it against the current view
      parameters (coordinates and precision) to validate compatibility.
      \item \emph{Load Imagina Orbit (Use Saved)...} (command): Loads a
      previously saved Imagina orbit file directly, using whatever parameters
      are stored in the file without matching against the current view.
    \end{itemize}

  Imagina is a separate Mandelbrot rendering application.  \FractalShark{}
  can import and export reference orbits in Imagina's file format, which is
  useful for cross-validating orbit computations between the two programs or
  for sharing precomputed orbits.

  \item \emph{Exit} (command): Exits the application.
\end{itemize}

\begin{figure}[!htbp]
  \centering
  \includegraphics[width=0.9\linewidth]{auto-renderer-artifact.png}
  \caption{Rendering artifacts arising from insufficient precision in the
  HDRx32 GPU renderer. Switching to HDRx64 or a CPU-based renderer resolves these
  artifacts.}
  \label{fig:auto-renderer-artifact}
\end{figure}

\subsection{Recalculation Commands}

Commands such as \emph{Recalculate} or \emph{Recalculate, Reuse Reference} trigger
an immediate re-render of the current view. These commands do not change any
persistent settings; they simply instruct the renderer to restart computation
using the current configuration. \emph{Recalculate, Reuse Reference} keeps the
existing high-precision reference orbit and only recomputes the per-pixel
perturbation pass, which is significantly faster when the view center has not
changed.

\subsection{Experimental and Debug Options}

Some menu entries expose experimental features, debugging aids, or partially
implemented functionality. These options may be unstable, incomplete, or
subject to removal. \FractalShark{} is an experimental research project rather than
a polished end-user application. Users should expect occasional crashes,
rendering failures, or unresponsive behavior when exercising less common menu
options.

\subsection{Practical Usage Notes}

\begin{itemize}
  \item There is currently no reliable way to cancel a long-running render once
  it has begun. If necessary, the application must be terminated externally.
  \item Some menu options may be temporarily disabled depending on the current
  render state or selected algorithm.
  \item If the application exits unexpectedly, a Windows minidump
  (\texttt{.dmp}) file may be generated in the same directory as the
  executable.  These dumps can be loaded in Visual Studio or WinDbg for
  post-mortem debugging.
\end{itemize}

Despite its rough edges, the popup menu provides a unified control surface for
exploring \FractalShark{}’s many rendering paths and numerical experiments, making
it the primary interface for both casual exploration and deep technical
investigation.


\section{Keyboard Shortcuts}
\label{sec:keyboard-shortcuts}

In addition to the popup menu, \FractalShark{} supports a set of single-key
shortcuts for navigation, recalculation, palette exploration, and tuning of
perturbation/linear-approximation parameters.  Most shortcuts operate relative
to the \emph{current mouse position} (converted into client coordinates at the
time the key is pressed), which makes them feel similar to context-menu actions.

Unless otherwise noted, shortcuts are \emph{case-sensitive} in the sense that
uppercase variants are invoked by holding \texttt{Shift}.  Many actions also
explicitly clear cached perturbation results prior to recomputation, trading
additional work for correctness or a ``clean'' run.

\paragraph{Navigation and view control.}
\begin{itemize}
  \item \texttt{z}: Zoom in by a fixed amount, centered at the mouse cursor.
  \item \texttt{Z} (\texttt{Shift+z}): Zoom out by a fixed amount, centered at the mouse cursor.
  \item \texttt{c}: Center the view at the mouse cursor (no forced reference recomputation).
  \item \texttt{C} (\texttt{Shift+c}): Center the view and clear all perturbation results first.
  \item \texttt{b}: Go back to the previous view.
  \item \texttt{r}: Square the current view (aspect normalization).
  \item \texttt{R} (\texttt{Shift+r}): Clear all perturbation results, then square the current view.
  \item Mouse \textbf{left-click and drag}: Zoom into a dragged rectangle.
        Holding \texttt{Shift} while dragging disables aspect-ratio enforcement.
\end{itemize}

\paragraph{Autozoom (experimental / buggy).}
\label{par:autozoom}

\begin{itemize}
  \item \texttt{a}: Autozoom using the ``Feature'' heuristic.  Forces the
        \texttt{GpuHDRx32PerturbedLAv2} render algorithm, centres on the
        perturbation reference point, computes a target zoom depth from the
        reference orbit's maximum radius, and zooms in incrementally while
        linearly interpolating the iteration count toward the reference orbit's
        target.
  \item \texttt{A} (\texttt{Shift+a}): Autozoom using the ``Default''
        heuristic.  First centres on the mouse position, then iteratively picks
        a zoom target via a weighted geometric mean of iteration counts
        (favouring high-iteration, centre-proximate pixels) and zooms by a
        factor of~3 each step.
\end{itemize}
Holding \texttt{Ctrl} may abort either autozoom variant (the abort mechanism is
not part of the key handler and is implemented elsewhere).

\paragraph{Recalculation and benchmarking.}
Several keys force a recalculation; uppercase forms often clear cached results
first.
\begin{itemize}
  \item \texttt{i}: Force recalculation and show the current position (copies details to clipboard).
  \item \texttt{I} (\texttt{Shift+i}): Clear \emph{medium-res} perturbation results, then recalc and show position.
  \item \texttt{o}: Force recalculation and show the current position.
  \item \texttt{O} (\texttt{Shift+o}): Clear \emph{all} perturbation results, then recalc and show position.
  \item \texttt{p}: Force recalculation and show the current position.
  \item \texttt{P} (\texttt{Shift+p}): Clear \emph{LA-only} perturbation results, then recalc and show position.
\end{itemize}
\emph{Note:} The in-application hotkey dialog groups \texttt{i/I}, \texttt{o/O},
\texttt{p/P}, and \texttt{r/R} under ``Recalculating and Benchmarking'' with
slightly different wording.  The authoritative behavior is the code path: these
keys primarily differ by which cached perturbation result class is cleared
before forcing recomputation.

\paragraph{Reference compression tuning.}
These keys adjust compression error exponents and then repaint, typically after
clearing cached perturbation results.
\begin{itemize}
  \item \texttt{e}: Clear all perturbation results, reset compression error exponents to defaults,
        and repaint.
  \item \texttt{q}: Clear all perturbation results; \emph{increase} intermediate compression error
        exponent by a step (more error, less memory).
  \item \texttt{Q} (\texttt{Shift+q}): Clear all perturbation results; \emph{decrease} intermediate compression
        error exponent by a step (less error, more memory).
  \item \texttt{w}: Clear all perturbation results; \emph{increase} low-error compression exponent by a step.
  \item \texttt{W} (\texttt{Shift+w}): Clear all perturbation results; \emph{decrease} low-error compression exponent by a step.
\end{itemize}
\emph{Practical interpretation:} increasing the error exponent generally favors
memory reduction and speed at the expense of fidelity in reconstructed reference
segments; decreasing it pushes toward higher fidelity and higher memory use.

\paragraph{Linear approximation parameter tuning (powers-of-two adjustments).}
These keys adjust linear-approximation thresholds and period-detection thresholds.
After changing parameters, \FractalShark{} clears LA-only perturbation results and
forces a recalc.
\begin{itemize}
  \item \texttt{h}: Increase LA threshold scale exponents (less accurate / faster per-pixel).
  \item \texttt{H} (\texttt{Shift+h}): Decrease LA threshold scale exponents (more accurate / slower per-pixel).
  \item \texttt{j}: Increase LA period-detection threshold exponents.
  \item \texttt{J} (\texttt{Shift+j}): Decrease LA period-detection threshold exponents.
\end{itemize}

\paragraph{Palettes.}
\begin{itemize}
  \item \texttt{d}: Advance to the next palette lookup-table depth and redraw.
  \item \texttt{D} (\texttt{Shift+d}): Create a new random palette, select it, and redraw.
  \item \texttt{t}: Increase auxiliary palette depth (cycling auxiliary palette behavior) and redraw.
  \item \texttt{T} (\texttt{Shift+t}): Decrease auxiliary palette depth and redraw.
\end{itemize}

\paragraph{Iteration count.}
These keys scale the maximum iteration count aggressively for quick exploration:
\begin{itemize}
  \item \texttt{=} : Multiply max iterations by 24.
  \item \texttt{-} : Multiply max iterations by $2/3$.
\end{itemize}

\paragraph{Window manipulation and menu invocation.}
\begin{itemize}
  \item \textbf{Right-click}: Open the popup menu at the cursor position.
  \item \textbf{Shift+F10 / Menu key}: Also opens the popup menu (Windows context-menu invocation).
  \item \textbf{Alt + left-click drag} (windowed mode): Drag the window by treating the client area as a caption.
  \item Mouse wheel: Zoom in/out (scroll up zooms in; scroll down zooms out).
\end{itemize}

\paragraph{Feature Finder.}
These keys invoke the periodic-point (feature) finder at the current mouse
position, or scan the entire viewport.  See \cref{sec:feature-finder}
for the algorithm details and \cref{sec:ui-popup-menu} for the
corresponding menu items under \emph{Navigate $\to$ Feature Finder}.
\begin{itemize}
  \item \texttt{n}: Find the nearest feature using direct iteration (single point at cursor).
  \item \texttt{N} (\texttt{Shift+n}): Scan the viewport on a $12\times12$ grid using direct iteration.
  \item \texttt{m}: Find the nearest feature using perturbation theory (single point at cursor).
  \item \texttt{M} (\texttt{Shift+m}): Scan the viewport on a $12\times12$ grid using perturbation theory.
  \item \texttt{,}: Find the nearest feature using linear approximation (single point at cursor).
  \item \texttt{<} (\texttt{Shift+,}): Scan the viewport on a $12\times12$ grid using linear approximation.
  \item \texttt{.}: Zoom to the most recently found feature (refines to high precision first if needed).
  \item \texttt{>} (\texttt{Shift+.}): Clear all found-feature overlays.
\end{itemize}

\paragraph{Responsiveness note.}
Many shortcuts trigger a recomputation or redraw immediately.  If \FractalShark{}
is executing a long-running render (especially a large GPU kernel), keyboard
input may appear delayed until the render completes.  In practice, the popup
menu and hotkeys are best used as \emph{configuration and relaunch} controls,
not as guaranteed real-time interaction during heavy computation.

\subsection{Commentary}

\begin{itemize}
\item One fun thing you can try is running with LAv2 + ``LA only''. This 
approach actually works pretty well once Linear Approximation kicks in at 
deeper zooms --- it gives you an idea of what the actual image should look like 
but is very fast, since it does no perturbation. The images it produces are not 
precise, and often leave out the fine detail; however, it's fun to play with 
when zooming in on a specific point.

\item The most interesting reference orbit calculation is at \texttt{
AddPerturbationReferencePointMT3}. It includes a ``bad'' calculation which is 
used for the ``scaled'' CUDA kernels. The multithreaded approach handily beats 
the single-threaded implementation on my CPU in all scenarios.

\item There are CPU renderers, but they were mostly to learn/debug more easily, 
and aren't optimized heavily. They're much easier to understand and reason 
about though.
\end{itemize}

\section{Kernel List}

This section simply lists all the CUDA kernels in \FractalShark{}.

\begin{itemize}

\item \textbf{Mandelbrot base kernels} (\cref{sec:mandel-base-kernels})
  \begin{itemize}
    \item \texttt{mandel\_1x\_float}
    \item \texttt{mandel\_1x\_double}
    \item \texttt{mandel\_2x\_float}
    \item \texttt{mandel\_2x\_double}
    \item \texttt{mandel\_4x\_float}
    \item \texttt{mandel\_4x\_double}
    \item \texttt{mandel\_hdr\_float}
  \end{itemize}

\item \textbf{Mandelbrot perturbation kernels}
  \begin{itemize}
    \item \texttt{mandel\_1x\_double\_perturb\_bla}
    \item \texttt{mandel\_1xHDR\_float\_perturb\_bla} (\cref{sec:bilinear-approx})
    \item \texttt{mandel\_1xHDR\_float\_perturb\_lav2} (\cref{sec:la-v2-perturb,sec:perturb-only})
  \end{itemize}

\item \textbf{High-precision reference kernels} (\cref{subsec:ref-orbit-gpu})
  \begin{itemize}
    \item \texttt{HpSharkReferenceGpuKernel}
    \item \texttt{HpSharkReferenceGpuLoop}
  \end{itemize}

\item \textbf{Utility kernels}
  \begin{itemize}
    \item \texttt{antialiasing\_kernel}
    \item \texttt{max\_reduce}
    \item \texttt{max\_kernel}
  \end{itemize}

\item \textbf{Addition kernels} (\cref{sec:hp-add})
  \begin{itemize}
    \item \texttt{AddKernel}
    \item \texttt{AddKernelTestLoop}
  \end{itemize}

\item \textbf{Multiplication (NTT) kernels} (\cref{sec:ntt-multiply})
  \begin{itemize}
    \item \texttt{MultiplyKernelNTT}
    \item \texttt{MultiplyKernelNTTTestLoop}
  \end{itemize}

\item \textbf{Disabled kernels}
  \begin{itemize}
    \item \texttt{mandel\_2x\_float\_perturb\_setup}
    \item \texttt{mandel\_2x\_float\_perturb}
  \end{itemize}

\end{itemize}

\subsection{Commentary}

\begin{itemize}

\item All the per-pixel CUDA kernels are in \texttt{gpu\_render.cu}. The two
most interesting are probably \texttt{mandel\_1xHDR\_float\_perturb\_bla} and
\texttt {mandel\_1xHDR\_float\_perturb\_lav2}, which are the ones I've spent the
most time on lately. For better performance at low zoom levels, you could look
at \texttt{mandel\_1x\_float\_perturb}, which leaves out linear approximation
and just does straight perturbation up to $\sim 10^{30}$, which corresponds with
the 32-bit float exponent range.

\item The \texttt{mandel\_1x\_float} (\cref{sec:mandel-1x-float}) is the classic 32-bit float Mandelbrot and 
is screaming fast on a GPU. This one is optimized with fused multiply-add for 
fun even though it's kind of useless because you can barely zoom in before you 
get pixellation.

\end{itemize}